\chapter{Conclusion and Future Work}

% The project's conclusions should list the key things that have been learnt as a consequence of engaging in your project work. For example, ``The use of overloading in C++ provides a very elegant mechanism for transparent parallelisation of sequential programs'', or ``The overheads of linear-time n-body algorithms makes them computationally less efficient than $O(n \log n)$ algorithms for systems with less than 100000 particles''. Avoid tedious personal reflections like ``I learned a lot about C++ programming...'', or ``Simulating colliding galaxies can be real fun...''. It is common to finish the report by listing ways in which the project can be taken further. This might, for example, be a plan for turning a piece of software or hardware into a marketable product, or a set of ideas for possibly turning your project into an MPhil or PhD.


% in hindsight, would have been great to have a camera with software to measure the rate of growth, but this was realised too late and could not be done

In hindsight, the rate of growth would have been a great measurement to take for the machine learning model. Unfortunately this would require a camera with software to recognise the plant in order to record its length. This realisation came too late in the project, after the majority of the experiments had been run. Another thing that could have been changed in the experiment is including night and day cycles. This would pair well with tracking the growth rate as we can see how well the cress grew during the day and night periods.

To further this project, having different vegetation along side a much larger period of time would allow for for thorough research in the different components of growth. With a larger data set, the population will be more accurately represented in the machine learning model.                                                                                                                         

